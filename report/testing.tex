\section{Testing}

To test and verify the correct operation of RDT, two test programs \code{RdtClient.c} and \code{RdtServer.c} were created to send an arbitrary file between two lab PCs using RDT. 

\subsection{Methodology}

To test that RDT packets were delivered reliably and in-order, a sufficiently large JPEG file (\emph{dog.jpg}) was used as test file. This provided visible feedback as to the integrity of the data received by \code{RdtServer}. This integrity was also verified by calculating SHA1 checksums of both the sent and received files. \code{slurpe-3} was used to provide an emulated path with loss, delay and restricted data rate to test the reliability of RDT in austere network conditions.

\subsection{Results}

From the results in Table \ref{tab:results} we can see the RDT performed as expected in a variety of austere network conditions.

\begin{center}
    \begin{table}[H]
    \begin{tabular}{|c|c|c|c|c|}
        \hline
        Scenario & In file & Out File & Match & Time \\
        \hline
        Control & \emph{dog.jpg} & \emph{dog-control.jpg} & \checkmark  & Success \\

        Delay & \emph{dog.jpg} & \emph{dog-d.jpg} & \checkmark  & Success \\

        Loss & \emph{dog.jpg} & \emph{dog-l.jpg} & \checkmark  & Success \\

        Constrained Rata & \emph{dog.jpg} & \emph{dog-cr.jpg} & \checkmark & Success \\

        Loss and Delay & \emph{dog.jpg} & \emph{dog-ld.jpg} & \checkmark & Success \\

        Loss and Constrained Rate & \emph{dog.jpg} & \emph{dog-lcr.jpg}  & \checkmark & Success \\

        Delay and Constrained Rate & \emph{dog.jpg} & \emph{dog-dcr.jpg} & \checkmark & Success \\

        Delay, Loss and Constrained Rate & \emph{dog.jpg} & \emph{dog-ldcr.jpg} & \checkmark & Success \\
        
        \hline
    \end{tabular}
    \caption{Results for varying network conditions}\label{tab:results}
    \end{table}
\end{center}