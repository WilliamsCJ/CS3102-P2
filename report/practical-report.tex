\documentclass[12pt]{article}
% \usepackage[utf8]{inputenc}
\usepackage{graphicx}
\usepackage{listings}

\setlength{\parindent}{0em}

\def\code#1{\texttt{#1}}

% FILL IN TITLE AND AUTHOR
\title{
{CS3102 P2: Practical Report}\\
{\large Reliable Data Transfer Using UDP}\\
{\includegraphics[width=80mm]{university-logo.png}}
}
\author{190010906}
\date{01 April 2022}

\begin{document}

\maketitle

\newpage

\section{Introduction}

This report cover the design and implementation of a connetion-oritented, reliable, unicast, transport protocol,built on top of UDP.

The protocol in question is called RDT - Reliable Data Transport

\section{Design}

This section will describe the design of the RDT protocol and the considerations that informed this design.

\subsection{Packet Structure}

RDT packets are constructed with the following structure:

\begin{figure}[h]
\begin{verbatim}
 0                   1                   2                   3
 0 1 2 3 4 5 6 7 8 9 0 1 2 3 4 5 6 7 8 9 0 1 2 3 4 5 6 7 8 9 0 1
+-+-+-+-+-+-+-+-+-+-+-+-+-+-+-+-+-+-+-+-+-+-+-+-+-+-+-+-+-+-+-+-+
|            Sequence           |              Type             |
+-+-+-+-+-+-+-+-+-+-+-+-+-+-+-+-+-+-+-+-+-+-+-+-+-+-+-+-+-+-+-+-+
|              Size             |            Checksum           |
+-+-+-+-+-+-+-+-+-+-+-+-+-+-+-+-+-+-+-+-+-+-+-+-+-+-+-+-+-+-+-+-+
\end{verbatim}
\caption{RDT Header}
\end{figure}

The \texttt{type} field contains one following values, denoting the type of the packet:

\begin{itemize}
    \item 0 - SYN.
    \item 1 - SYN ACK
    \item 2 - DATA
    \item 3 - DATA ACK
    \item 4 - FIN
    \item 5 - FIN ACK
\end{itemize}

This approach was chosen over a flag-based approach, as it makes it easier to check packet type and there were only a small number of types to define given the simple nature of the protocol.

32 bit values for were chosen for sequence and size as long data type often used for file sizes, and subsequently need to have 32 bit for sequence as product of buffer size.

200 ms is used initially for the handshake RTO, as packet is only 12 bytes.

\subsection{Finite State Machine}

\section{Testing}

To test and verify the correct operation of RDT, two test programs \code{RdtClient.c} and \code{RdtServer.c} were created to send an arbitrary file between two lab PCs using RDT. 

\subsection{Methodology}

To test that RDT packets were delivered reliably and in-order, a sufficiently large JPEG file (\emph{dog.jpg}) was used as test file. This provided visible feedback as to the integrity of the data received by \code{RdtServer}. This integrity was also verified by calculating SHA1 checksums of both the sent and received files. \code{slurpe-3} was used to provide an emulated path with loss, delay and restricted data rate to test the reliability of RDT in austere network conditions.

\subsection{Results}

From the results in Table \ref{tab:results} we can see the RDT performed as expected in a variety of austere network conditions.

\begin{center}
    \begin{table}[H]
    \begin{tabular}{|c|c|c|c|c|}
        \hline
        Scenario & In file & Out File & Match & Time \\
        \hline
        Control & \emph{dog.jpg} & \emph{dog-control.jpg} & \checkmark  & Success \\

        Delay & \emph{dog.jpg} & \emph{dog-d.jpg} & \checkmark  & Success \\

        Loss & \emph{dog.jpg} & \emph{dog-l.jpg} & \checkmark  & Success \\

        Constrained Rata & \emph{dog.jpg} & \emph{dog-cr.jpg} & \checkmark & Success \\

        Loss and Delay & \emph{dog.jpg} & \emph{dog-ld.jpg} & \checkmark & Success \\

        Loss and Constrained Rate & \emph{dog.jpg} & \emph{dog-lcr.jpg}  & \checkmark & Success \\

        Delay and Constrained Rate & \emph{dog.jpg} & \emph{dog-dcr.jpg} & \checkmark & Success \\

        Delay, Loss and Constrained Rate & \emph{dog.jpg} & \emph{dog-ldcr.jpg} & \checkmark & Success \\
        
        \hline
    \end{tabular}
    \caption{Results for varying network conditions}\label{tab:results}
    \end{table}
\end{center}

\section{Analysis}

\subsection{Performance in Different Network Scenarios}

\begin{figure}[H]
\begin{center}
    \includegraphics[width=100mm]{images/performance-network-scenarios.png}
\end{center}
\caption{Total Transmission Time in Different Network Scenarios for 230 Kb file.}
\end{figure}

\begin{itemize}
    \item Bandwidth utilization
\end{itemize}

\subsection{RDT Packet Data Size}


\subsection{Wastage due to Control Information}

\begin{align*}
    \rho = \frac{i - c}{i + a} = \frac{1312 - 12}{1312 + 12} = 0.982 \text{\:(3 s.f.)}
\end{align*}

\end{document}
